\documentclass[12pt]{article}
\usepackage{graphicx, subfigure, float, amsmath, amssymb, color}
\usepackage[margin = 1.0 in]{geometry}
\usepackage[backend=biber,firstinits=true,sorting=none]{biblatex}
\addbibresource{article.bib}

\title{Accessing biological data with semantic web technologies}
\author{Egon L.\ Willighagen$^{1,2}$}
\begin{document}

\date{\today}
\maketitle

\noindent
$^1$ Department of Bioinformatics - BiGCaT, NUTRIM, Maastricht University, Maastricht, The Netherlands \\
$^2$ Department of Molecular Toxicology, IMM, Karolinska Institutet, Stockholm, Sweden

\bigskip
\noindent
$^*$Corresponding author\\
$\phantom{^*}$Email: egon.willighagen@maastrichtuniversity.nl\\
$\phantom{^*}$Twitter: @egonwillighagen\\
$\phantom{^*}$ORCID: http://orcid.org/0000-0001-7542-0286

\bigskip
\noindent
Manuscript type: research article

\bigskip
\noindent Keywords: semantic web, workflow

\newpage
\begin{abstract}
\end{abstract}

\section{Introduction}

Semantic Web technologies are finding their way into biological
databases~\cite{Belleau2008,Chen2010,Samwald2011,Williams2012,Willighagen2013}. The recent launch of the
Resource Description Framework (RDF) services at the European Bioinformatics Institute is a good example
of the impact it has on the field, despite the adoption taking place for longer already~\cite{EBIRDFPlatform}.
At the same time, most new databases do not yet use semantic web technologies, which may be partly
caused by the lack of availability of tools to handle RDF data.

RDF allows the combination of a machine readable framework with the use of ontologies that allow
sharing data and the meaning in the some data exchange format. This is enabled by a set of open
specifications, such as the underlying RDF, RDF Schema, the Web Ontology Language, and serialization formats
such as RDF/XML, N-TRIPLES, Notation3 (N3), and Turtle. Crucial, however, seems the specification of the SPARQL
query language that allow extracting data from local or remote RDF triple stores~\cite{Seaborne2009}. In fact,
federated SPARQL queries allow extracting data from multiple triple stores at the same time.

At the same time, R has established itself as a key tool for statistics, particularly in the biological
sciences. Many packages have been developed for this domain and are available from CRAN and Bioconductor,
allowing to combine various bioinformatics and data analysis approaches. These packages include,
for example, biomaRt, which have the purpose of extracting information from a remote database.

We here introduce two packages, rrdflibs and rrdf, for the statistics environment R, based on the Apache
Jena library and the RJava interface to Java~\cite{RMan,Jena,RJava}. Compared to the existing SPARQL
package, this package adds general RDF data handling~\cite{VanHage2013}. Using this package we show how
data aggregation can be performed from remote SPARQL end points, and how local data in other formats
can be converted into RDF to querying with SPARQL.

\section{Materials \& Methods}

The rrdf packages provides RDF and SPARQL functionality by exposing that functionality from the
Apache Jena library~\cite{Jena}. The rrdflibs package contains the Apache Jena libraries and
nothing more. This package is close to 7~MiB large but does not change frequently. The rrdf
package contains the R functions that wrap around the Jena functionality and convert data structures
where needed. This package is about 112~kiB but changes more frequently. Using this approach,
the Comprehensive R Archive Network (CRAN) server has the least amount of download throughput.
The Jena functionality is accessible using the rJava library~\cite{RJava} which handles loading
of the Java archives and provides methods to instantiate classes and call methods.

The source code of the package is available from GitHub at \url{http://github.com/egonw/rrdf/} and
binary packages are available from CRAN at \url{http://cran.r-project.org/web/packages/rrdf/}.
The first patch is from March 2011 while the most recent patches are from late 2013 when the
rrdflibs package was updated for Apache Jena 2.11.

\section{Results and Discussion}

The rrdf package provides basic and less basic functionality for aggregation and analysis of RDF data.
The package can be install and loaded with these commands:

\begin{verbatim}
install.packages(c("rrdf", "rrdflibs"))
library(rrdf)
\end{verbatim}

\subsection{Triple stores}

To handle triples, we first need a triple store. At this moment only in-memory stores are
supported, though Jena's TDB also provides on-disk triple stores; this puts limitations to
the amount of data you can analyze in one study.
The package supports two kinds of stores, one that has minimal ontology support, and one
basically just handles triples. Both can be created with the \textit{new.rdf} command:

\begin{verbatim}
ontStore = new.rdf()
store = new.rdf(ontology=FALSE)
\end{verbatim}

When data is not read from a file or downloaded, but created from data in another format,
such as an R matrix, individual triples can be added to a store. This works for both
object properties and data properties. The former option links a subject to an object
resource, while the latter links a subject to a literal value:

\begin{verbatim}
add.triple(store,
  subject="http://example.org/Subject",
  predicate="http://example.org/Predicate",
  object="http://example.org/Object"
)
add.data.triple(store,
  subject="http://example.org/Subject",
  predicate="http://example.org/Predicate",
  data="Literal value"
)
\end{verbatim}

Data can be loaded from file by providing a file name and a format ("RDF/XML", "TURTLE",
"N-TRIPLES", and "N3"):

\begin{verbatim}
store = load.rdf("file.n3", format="N3")
\end{verbatim}

\subsection{Example}

As an example, we will here create RDF for a small data set with information on five single nucleotide
polymorphisms (SNPs) in the BRCA1 gene, extracted with biomaRt~\cite{Durinck2005}:

\begin{verbatim}
library(biomaRt)
mart = biomaNoy2009Rt::useMart(biomart="snp", dataset="hsapiens_snp")
brca1 = c("rs16940","rs16941", "rs16942", "rs799916", "rs799917")
attribs = c(
  "refsnp_id", "chr_name", "chrom_start",
  "validated", "pmid_20137", "allele"
)
data = biomaRt::getBM(
  attributes=attribs, filters=c("snp_filter"), values=brca1, mart=mart
)
\end{verbatim}

Following the guidelines for generating RDF outlined by Open PHACTS~\cite{Haupt2013,Williams2012}, we identify concepts
and matching ontology terms, using BioPortal~\cite{Noy2009}: SNP, gene, chromosome,
article, and allele. Additionally, there are the following properties: a SNP identifier, a PubMed identifier,
a chromosome number, a chromosomal sequence position, and a list of validations (e.g. HapMap, 1000Genomes).
The latter will not be further semantically specified, but just be converted into a RDF literal as it is
provided by the BioMart.

\begin{table}
\caption{A list of ontology terms for the concepts in the BRCA1 SNP example. Ontology acronyms and the prefixes
used are explained in the text.}
\begin{center}
\begin{tabular}{l|l|l|l}
 \textbf{concept} & \textbf{ontology} & \textbf{term} & \textbf{URI} \\
\hline
 SNP & SIO & snp & sio:SIO\_010027 \\
 gene & SO & gene & obo:SO\_0000704 \\
 chromosome & GO & chromosome & obo:GO\_0005694 \\
 article &  BIBO & article & bibo:Article \\
 allele & SO & allele & obo:SO\_0001023 \\
 SNP identifier & TMO & snp identifier & tmo:TMO\_0161 \\
 PubMed identifier & CHEMINF & PubMed identifier & sio:CHEMINF\_000302\\
 chromosome number & TMO & chromosome nuber & tmo:TMO\_0157 \\
 chromosomal sequence position & TMO & chromosomal sequence position & tmo:TMO\_0122 \\
\end{tabular}CiTO
\end{center}
\end{table}

Taking advantage of existing ontologies. The choice is somewhat arbitrary and no ontological
analysis has been done, e.g. on specified domain and ranges of predicates and implied
properties of classes. The following ontologies are used:
SIO is the Semanticscience Integrated Ontology; TMO is the Translational Medicine Ontology;
SO is the Sequence Types and Features Ontology; GO is the Gene Ontology;
BIBO is the Bibliographic Ontology; CiTO is the Citation Typing Ontology;
... Prefixes used in the term Uniform Resource Identifiers (URIs) include
\textit{sio} for \url{http://semanticscience.org/resource/},
\textit{tmo} for \url{http://www.w3.org/2001/sw/hcls/ns/transmed/},
\textit{obo} for \url{http://purl.obolibrary.org/obo/},
\textit{bibo} for \url{http://purl.org/ontology/bibo/},
\textit{cito} for \url{} ...

Using these ontology terms we can define the translations of the BioMart-extracted
data into RDF:

\begin{verbatim}
snpStore = new.rdf(ontology=FALSE)
rdfTypePred = "http://www.w3.org/1999/02/22-rdf-syntax-ns#type"
snpClass = "http://semanticscience.org/resource/SIO_010027"
createEntry <- function(row) {
  snpID = row[1]
  snpSubject = paste("http://example.org/snp/", snpID, sep="")
  add.triple(snpStore,
    subject=snpSubject, predicate=rdfTypePred, object=snpClass
  )
}
\end{verbatim}

We can then just iterate over the actual data with:

\begin{verbatim}
apply(data, MARGIN=1, FUN=createEntry(row) )
\end{verbatim}

\subsection{SPARQL}

Local (in-memory) triple stores can be queried with the \emph{sparql.rdf} method by providing
it with the store to query and a string with the SPARQL query. For example, to get all resource
types, we can use this:

\begin{verbatim}
sparql.rdf(store,
  paste(
    "SELECT DISTINCT ?type WHERE {",
    " [] a ?type ",
    "}"
  )
) 
\end{verbatim}

Remote SPARQL end point can be queried in a similar fashion: we replace the triple store with the
URL pointing to the SPARQL end point. For example, if we want to
extract all properties of the CHEMBL615603 assay from ChEMBL~\cite{Gaulton2012,Willighagen2013}
we can use the following remote query:

\begin{verbatim}
results = sparql.remote(
  "http://rdf.farmbio.uu.se/chembl/sparql",
  paste(
    "SELECT DISTINCT ?predicate ?object WHERE {",
    " ?assay <http://www.w3.org/2000/01/rdf-schema#label> \"CHEMBL615603\" ;",
    "        ?predicate ?object .",
    "}"
  )
) 
\end{verbatim}

Jena parses the SPARQL query too, which requires the query to be valid
against both Jena and the SPARQL end point. Because the SPARQL specification has tool specif
extensions and that not all tools support the full of SPARQL 1.1, it can sometimes be tricky
to find the mutually support SPARQL command subset. The rrdf package also allows to bypass
Jena and query the end point directly by using \emph{sparql.remote(..., jena=FALSE)},
removing the problem and allowing you to use extensions of the end point.

The results object is a matrix with the variable names from the SPARQL query as column names.
This allows us to get a predicates with:

\begin{verbatim}
results[,"predicate"]
\end{verbatim}

\subsection{Example}

...

%\subsection{Example: BioPortal}
%
%In the BRCA1 SNP example, we found ontology terms for concepts using BioPortal. Now, at the
%time of writing BioPortal has an experimental SPARQL end point which allows us to find terms
%in this way too. For example, to find terms related to SNPs, we can run this code in R:
%
%\begin{verbatim}
%findSNPterms = paste(
%  "PREFIX rdfs: <http://www.w3.org/2000/01/rdf-schema#>",
%  "SELECT DISTINCT ?term ?label WHERE {",
%  " ?term rdfs:label ?label .",
%  "FILTER (CONTAINS (LCASE(str(?label)), \"snp\"))",
%  "}"
%)
%sparql.remote(
%  endpoint="http://sparql.bioontology.org/sparql",
%  sparql=findSNPterms, jena=FALSE
%)
%\end{verbatim}

\section{Conclusions}

The rrdflibs and rrdf packages bring semantic web technologies to R statistical environment.
While only a subset of Apache Jena is currently exposed, it provides key methods to deal with
RDF data and resources.

\section{Acknowledgements}

The people who have provided feedback on the rrdf package are kindly thanked for their efforts in
getting into contact with me.

\printbibliography

\end{document}

\documentclass[12pt]{article}
\usepackage{graphicx, subfigure, float, amsmath, amssymb, color}
\usepackage[margin = 1.0 in]{geometry}
\usepackage[backend=biber,firstinits=true,sorting=none]{biblatex}
\addbibresource{article.bib}

\title{Accessing biological data with semantic web technologies}
\author{Egon L.\ Willighagen$^1,2$}
\begin{document}

\date{\today}
\maketitle

\noindent
$^1$ Department of Bioinformatics - BiGCaT, NUTRIM, Maastricht University, Maastricht, The Netherlands
$^2$ Department of Molecular Toxicology, IMM, Karolinska Institutet, Stockholm, Sweden

\bigskip
\noindent
$^*$Corresponding author\\
$\phantom{^*}$Email: egon.willighagen@maastrichtuniversity.nl\\
$\phantom{^*}$Twitter: @egonwillighagen\\
$\phantom{^*}$ORCID: http://orcid.org/0000-0001-7542-0286

\bigskip
\noindent
Manuscript type: research article

\bigskip
\noindent Keywords: semantic web, workflow

\newpage
\begin{abstract}
\end{abstract}

\section{Introduction}

Semantic Web technologies are finding their way into biological
databases~\cite{Belleau2008,Chen2010,Samwald2011,Williams2012,Willighagen2013}. The recent launch of the
Resource Description Framework (RDF) services at the European Bioinformatics Institute is a good example
of the impact it has on the field, despite the adoption taking place for longer already~\cite{EBIRDFPlatform}.
At the same time, most new databases do not yet use semantic web technologies, which may be partly
caused by the lack of availability of tools to handle RDF data.

RDF allows the combination of a machine readable framework with the use of ontologies that allow
sharing data and the meaning in the some data exchange format. This is enabled by a set of open
specifications, such as the underlying RDF, RDF Schema, the Web Ontology Language, and serialization formats
such as RDF/XML, N-TRIPLES, and Turtle. Crucial, however, seems the specification of the SPARQL
query language that allow extracting data from local or remote RDF triple stores~\cite{Seaborne2009}. In fact,
federated SPARQL queries allow extracting data from multiple triple stores at the same time.

At the same time, R has established itself as a key tool for statistics, particularly in the biological
sciences. Many packages have been developed for this domain and are available from CRAN and Bioconductor,
allowing to combine various bioinformatics and data analysis approaches. These packages include,
for example, biomaRt, which have the purpose of extracting information from a remote database.

We here introduce two packages, rrdflibs and rrdf, for the statistics environment R, based on the Apache
Jena library and the RJava interface to Java~\cite{RMan,Jena,RJava}. Using this package we show how
data aggregation can be performed from remote SPARQL end points, and how local data in other formats
can be converted into RDF to querying with SPARQL.

\section{Materials \& Methods}

The rrdf packages provides RDF and SPARQL functionality by exposing that functionality from the
Apache Jena library~\cite{Jena}. The rrdflibs package contains the Apache Jena libraries and
nothing more. This package is close to 7~MiB large but does not change frequently. The rrdf
package contains the R functions that wrap around the Jena functionality and convert data structures
where needed. This package is about 112~kiB but changes more frequently. Using this approach,
the Comprehensive R Archive Network (CRAN) server has the least amount of download throughput.
The Jena functionality is accessible using the rJava library~\cite{RJava} which handles loading
of the Java archives and provides methods to instantiate classes and call methods.

The source code of the package is available from GitHub at \url{http://github.com/egonw/rrdf/} and
binary packages are available from CRAN at \url{http://cran.r-project.org/web/packages/rrdf/}.
The first patch is from March 2011 while the most recent patches are from late 2013 when the
rrdflibs package was updated for Apache Jena 2.11.

\section{Results and Discussion}

The rrdf package provides basic and less basic functionality for aggregation and analysis of RDF data.
The package can be install and loaded with these commands:

\begin{verbatim}
install.packages(c("rrdf", "rrdflibs"))
library(rrdf)
\end{verbatim}

\subsection{Triple stores}



\subsection{SPARQL}


\section{Conclusions}

\section{Acknowledgements}

The people who have provided feedback on the rrdf package are kindly thanked for their efforts in
getting into contact with me.

\printbibliography

\end{document}
